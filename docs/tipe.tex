\documentclass[a4paper, 11pt]{article}
\usepackage[utf8]{inputenc} 
\usepackage[T1]{fontenc}
\usepackage{lmodern}
\usepackage[french]{babel}

\usepackage{amsmath}
\usepackage{amssymb}
\usepackage{amsthm}
\usepackage{mathtools}

\usepackage{graphicx}
\usepackage{tikz}
\usetikzlibrary{calc,trees,positioning,arrows,chains,shapes.geometric,%
    decorations.pathreplacing,decorations.pathmorphing,shapes,%
    matrix,shapes.symbols}

\theoremstyle{definition}
\newtheorem{definition}{Definition}[section]

\begin{document}

\title{Algorithmique des jeux de stratégie}
\author{Dimitri \bsc{Cocheril-Crèvecoeur}}


\section{La recherche hiérarchique par portfolio}
Un problème fréquent des gens de stratégie est la taille des cartes, rendant trop
lourds les algorithmes classiques d'ia de jeu comme l'algorithme alpha-béta. 
Nous étudierons ici une évolution de la recherche gourmande par portfolio : la
recherche hiéarchique par portfolio (HPS) a été inventée.\\
Les éléments qui forment une HPS sont :
\begin{description}
    \item[État] $s$ contenant toutes les informations du jeu nécessaires
    \item[Mouvement] $m= ( a_1, a_2, ..., a_k )$ séquence d'actions $a_i$
    \item[Joueur] fonction $p(s) = m$ prenant l'état du jeu et renvoyant les 
    actions choisies par l'alghorithme
    \item[Jeu] fonction $g(s, p_1, p_2, ..., p_k) = s'$ prenant l'état du jeu et
    les fonctions des joueurs et effectuant les actions
    \item[Joueur partiel] fonction $pp(s) = m$ joueur pariel \\
    elle est similaire à la fonction joueur $p$ mais au lieu de calculer les actions
    de toutes les unitées elle calcule celle d'une seule unitée.
    \item[Portfolio] $P = (pp_1, pp_2, ..., pp_k)$ ensemble des joueurs partiels
\end{description}

\end{document}
