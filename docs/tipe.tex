\documentclass[a4paper, 11pt]{article}
\usepackage[utf8]{inputenc} 
\usepackage[T1]{fontenc}
\usepackage{lmodern}
\usepackage[french]{babel}

\usepackage{amsmath}
\usepackage{amssymb}
\usepackage{amsthm}
\usepackage{mathtools}

\usepackage{graphicx}
\usepackage{tikz}
\usetikzlibrary{calc,trees,positioning,arrows,chains,shapes.geometric,%
    decorations.pathreplacing,decorations.pathmorphing,shapes,%
    matrix,shapes.symbols}

\theoremstyle{definition}
\newtheorem{definition}{Definition}[section]

\begin{document}

\title{Algorithmique des jeux de stratégie}
\author{Dimitri \bsc{Cocheril-Crèvecoeur}}

\section*{Définitions}
\begin{description}
    \item[Action] On définit deux actions : 
    \begin{itemize}
        \item Bouger l'unitée $a$ à la position $p$
        \item Attaquer l'unitée $b$ avec l'unité $a$
    \end{itemize} 
\end{description}

\section{Upper Confidence bounds applied to Trees}
Évolution de algorithme de recherche arborescente de Monte-Carlo.
L'algorithme recherche dans l'arbre des possibles. La formule du UCT est :
$$UCT = \overline{X_j} + C_p \sqrt{\frac{2 \ln(n)}{n_j}} $$ 
avec :
\begin{itemize}
    \item $n$ le nombre de fois que le parent a été visité
    \item $n_j$ le nombre de fois que l'enfant a été visité
    \item $\overline{X_j}$ le ratio de victoire : $$\overline{X_j} = \frac{\text{victoires} + \frac{\text{égalités}}{2}}{\text{victoires} + \text{égalités} + \text{défaites}}$$
    \item $C$ une constante permettant d'ajuster le nombre d'exploration de chaque noed
\end{itemize}
L'algorithme classique UCT n'est pas adapté aux jeux en temps réél. Ainsi, une évolution,
le UCT Considering Durations (UCTCD). 
L'algorithme se déroule en quatres étapes :
\\ \begin{tikzpicture}
    [
        sibling distance=7em,
        every node/.style = {shape=rectangle, rounded corners, draw, align=center}
    ]
    \node {Formulas}
      child { node {single-line} };
  \end{tikzpicture}

\section{La recherche hiérarchique par portfolio}
Un problème fréquent des gens de stratégie est la taille des cartes, rendant trop
lourds les algorithmes classiques d'ia de jeu comme l'algorithme alpha-béta. 
En effet, le nombre d'actions possibles que l'algorithme UCT considère est $L^U$
avec $L$ le nombre d'action possible moyen et $U$ le nombre d'unités possibles.
Nous étudierons ici une évolution de la recherche glouton par portfolio : la
recherche hiéarchique par portfolio (HPS) a été inventée.\\
Les éléments qui forment une HPS sont :
\begin{description}
    \item[État] $s$ contenant toutes les informations du jeu nécessaires
    \item[Mouvement] $m= ( a_1, a_2, ..., a_k )$ séquence d'actions $a_i$
    \item[Joueur] fonction $p(s) = m$ prenant l'état du jeu et renvoyant les 
    actions choisies par l'alghorithme
    \item[Jeu] fonction $g(s, p_1, p_2, ..., p_k) = s'$ prenant l'état du jeu et
    les fonctions des joueurs et effectuant les actions
    \item[Joueur partiel] fonction $pp(s) = m$ joueur pariel \\
    elle est similaire à la fonction joueur $p$ mais au lieu de calculer les actions
    de toutes les unitées elle calcule celle d'une seule unitée.
    \item[Portfolio] $P = (pp_1, pp_2, ..., pp_k)$ ensemble des joueurs partiels
\end{description}
L'algorithme HPS permet alors de changer un nombre de combibaison exponentiel
d'action en nombre linéaire d'action.

\end{document}
