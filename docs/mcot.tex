\documentclass[a4paper, 12pt]{article}
\usepackage[utf8]{inputenc} 
\usepackage[T1]{fontenc}
\usepackage{geometry}
\usepackage{lmodern}
\usepackage[french]{babel}
\usepackage{hyperref}
\usepackage{notoccite}

\usepackage{amsmath}
\usepackage{amssymb}
\usepackage{amsthm}
\usepackage{mathtools}
\usepackage{listings}
\usepackage{multicol}
\usepackage{comment}
\usepackage{enumerate}
\usepackage{enumitem}

\addto\captionsfrench{%
  \renewcommand*{\refname}{Références bibliographiques}%
}

\setlist[enumerate]{wide=0pt, leftmargin=15pt, labelwidth=15pt, align=left}

\geometry{
a4paper,
total={170mm,257mm},
left=20mm,
top=20mm,
}

\setlength{\parindent}{0cm}

\begin{document}

\title{Étude de quelques algorithmes de joueurs artificiels participants à des jeux de stratégie en temps réel}
\author{Dimitri \bsc{Cocheril-Crèvecœur}}
\date{2023-2024}

\maketitle

\section*{Motivations}
J'ai en 2020 entrepris de coder en C++ le jeu de stratégie Stellaris pour une
architecture de calculatrice. J'avais abandonné l'idée d'incorporer des joueurs
artificiels. Cette année j'ai donc été motivé pour m'intéresser à la réalisation
de joueur artificiel, cette fois-ci sur un jeu plus documenté, StarCraft.

\subsection*{Lien thème}

\section*{Positionnement thématique}
\emph{INFORMATIQUE (informatique pratique et théorique)} 

\section*{Mots clés}

\begin{tabular}{@{}ll}
    \textbf{Mots clés (en français)} & \textbf{Mots clés (en anglais)}  \\
    \begin{tabular}{@{}l@{}}Recherche Arborescente \\ Monte-Carlo\end{tabular} & Monte Carlo Tree Search (MCTS) \\
    Stratégie en temps réel & Real-time strategy (RTS) \\
    Joueur artificiel & Bot/agent \\
    Modèle de combat & Combat model \\
\end{tabular}

\section*{Bibliographie commentée}
StarCraft, un jeu de stratégie en temps réel (RTS), sous-genre des jeux de stratégie, 
est depuis plusieurs années source d'avancées dans le domaine de l'IA du fait de son 
fonctionnement complexe. \cite{stateoftheart} 
Une partie peut contenir entre 50 et 400 unités sur une carte de taille $128\times 128$. 
\cite{survey} Ces unités peuvent faire un grand nombre d'actions 24 fois par secondes.

Les meilleurs bots actuels sont faits avec des modèles IA neuronaux.
Le but de ce TIPE est d'étudier la conception d'algorithmes de joueurs artificiels plus classiques.

J'ai ainsi mis en place un joueur utilisant l'algorithme Monte-Carlo Tree Search.
Étant particulièrement gourmand en ressources et son exécution étant limitée dans le temps,
un certain travail d'optimisation a été nécessaire ainsi qu'une l'implémentation multithreadée.
Le choix d'une fonction d'évaluation est également un challenge.\cite{MCTSRTS}\cite{MCTStactical}

Les algorithmes classiques les plus performants ont fait le choix de monter en abstraction.
J'essaierai de le faire si le temps me le permet, en groupant les unités en armées grâce à l'algorithme DBSCAN.
\cite{dbscan}\cite{combatmodel}


\section*{Problématique retenue}
Quel algorithme est le plus efficace en stratégie de combat sur un jeu de stratégie en temps réel ?

\section*{Objectifs du TIPE du candidat}
\begin{itemize}
 \item Comparer différentes stratégies d'agents artificiels
\end{itemize}

\bibliographystyle{unsrt}

\bibliography{bibli}

\section*{DOT}
\begin{enumerate}[label=\textbf{[\arabic*]}]
    \item avril 2023 : Recherches sur les études déjà réalisées et les codes déjà produits en lien avec le sujet.
    \item mai/juin 2023 : Codage du moteur du jeu en C++, inspiré de la réelle API de \emph{StarCraft}.
    \item juillet 2023 : Gros travail d'optimisation sur le moteur, principalement du pathfinding.
    \item septembre/octobre 2023 : Création et tests du MCTS, puis son perfectionnement, avec multithreading du moteur.
    \item novembre 2023 - janvier 2024 :
\end{enumerate}

\end{document}
