\documentclass[a4paper, 11pt]{article}
\usepackage[utf8]{inputenc} 
\usepackage[T1]{fontenc}
\usepackage{lmodern}
\usepackage[french]{babel}
\usepackage{hyperref}

\usepackage{amsmath}
\usepackage{amssymb}
\usepackage{amsthm}
\usepackage{mathtools}
\usepackage{listings}


\begin{document}

\title{Étude de joueur artificiel de jeu de grande stratégie}
\author{Dimitri \bsc{Cocheril-Crèvecœur}}
\date{2023-2024}

\maketitle

\section*{Introduction}
J'étudie un modèle de joueur artificiel de jeu de grande stratégie (aussi appelé RTS).
Les RTS sont un genre de jeu de stratégie, sans meilleure stratégie, en temps réel, avec un grand espace d'action.
J'ai décidé de travailler principalement sur l'application du Monte Carlo Tree Search (MCTS) au problème.
Le MCTS est souvent utilisé dans des jeux au tour par tour, par exemple le GO ou les échec.

\bibliographystyle{plain}
\nocite{*}
\bibliography{bibli}

\end{document}
