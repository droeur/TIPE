\documentclass[a4paper, 11pt]{article}
\usepackage[utf8]{inputenc} 
\usepackage[T1]{fontenc}
\usepackage{lmodern}
\usepackage[french]{babel}
\usepackage{hyperref}

\usepackage{amsmath}
\usepackage{amssymb}
\usepackage{amsthm}
\usepackage{mathtools}
\usepackage{listings}
\usepackage{multicol}
\usepackage{comment}
\usepackage{enumerate}

\addto\captionsfrench{%
  \renewcommand*{\refname}{Références bibliographiques}%
}

\begin{document}

\title{Étude de quelques algorithmes de joueurs artificiels participants à des jeux de stratégie en temps réel}
\author{Dimitri \bsc{Cocheril-Crèvecœur}}
\date{2023-2024}

\maketitle

\section*{Motivations}
J'ai en 2020 entrepris de coder en C++ le jeu de grande stratégie \emph{Stellaris} en 24 bit pour une architecture de processeur de calculatrice (ez80).
J'avais abandonné l'idée d'incorporer des joueurs artificiels à l'époque. 
Cette année j'étais donc motivé pour m'intéresser à la réalisation de joueur artificiel, cette fois ci sur un jeu plus documenté, \emph{StarCraft}.

\section*{Positionnement thématique}
\emph{INFORMATIQUE (informatique pratique et théorique)} 

\section*{Mots clés}
\begin{tabular}{ll}
    \textbf{Mots clés} (Français) & \textbf{Mots clés} (Anglais) \\
    \begin{tabular}{@{}l@{}}Recherche Arborescente \\ Monte-Carlo\end{tabular} & Monte Carlo Tree Search (MCTS) \\
    Stratégie en temps réel & Real-time strategy (RTS) \\
    Joueur artificiel & Bot/agent \\
    Modèle de combat & Combat model \\
\end{tabular}

\section*{Bibliographie commentée}
StarCraft, un jeu de stratégie en temps réel (RTS), sous-genre des jeux de stratégie, 
est depuis plusieurs années source d'avancées dans le domaine de l'IA à cause de son 
fonctionnement complexe. \cite{stateoftheart} 
Une partie peut contenir entre 50 et 400 unités sur une carte de taille $128\times 128$. 
\cite{survey} Ces unités peuvent faire un certain ensemble d'actions avec une certaine 
latence 24 fois par secondes.

Les meilleurs bots actuels sont faits avec des modèles IA neuronales.
Le but de ce TIPE est d'étudier la conception d'algorithmes de joueurs artificiels plus classiques.

J'ai ainsi mis en place un joueur utilisant l'algorithme Monte-Carlo Tree Search.
Étant particuliérement gourmand en ressources, et son éxécution étant limitée dans le temps,
un certain travail d'optimisation a été nécessaire ainsi que l'implémentation d'un multithreading.
Le choix d'une fonction d'évaluation est également un challenge fréquent.\cite{MCTSRTS}\cite{MCTStactical}

Les algorithmes classiques les plus performants ont fait le choix de monter en abstraction.
J'essaierai de le faire si le temps me le permet, en groupant les unités en armées grâce à l'algorithme DBSCAN.
\cite{dbscan}\cite{combatmodel}


\section*{Problématique retenue}
Quel algorithme est le plus efficace en stratégie de combat sur un RTS ?

\section*{Objectifs du TIPE du candidat}
Pour réaliser le TIPE, il a d'abord fallu développer un moteur de jeu typé \emph{StarCraft} et une api.
J'entreprends ensuite la création de joueurs artificiels à différents niveaux d'abstraction :
un joueur artificiel utilisant l'algorithme MCTS multithreadé, et un joueur artificiel utilisant une technique de recherche par portfolio.
Nous nous intéressons particulièrement aux phases de combat, et n'abordons l'aspect gestion de ressources que marginalement, en seconde partie d'étude.

\section*{Abstract}

Real-time strategy games are challenging for the creation of bots due to their real-time and large-scale aspects.
A large community of players and researchers is actively trying to create the perfect bot through different techniques.
I study here two scripts that have already proven to be good enough to beat some human players, but not the best world champions.


\bibliographystyle{plain}

\bibliography{bibli}

\section*{DOT}
\begin{enumerate}[{[1]}]
    \item avril 2023 : Recherches sur les études déjà réalisées et les codes déjà produits en lien avec le sujet.
    \item mai/juin 2023 : Codage du moteur du jeu en C++, inspiré de la réelle API de \emph{StarCraft}.
    \item juillet 2023 : Gros travail d'optimisation sur le moteur, principalement du pathfinding.
    \item septembre/octobre 2023 : Création et tests du MCTS, puis son perfectionnement, avec multithreading du moteur.
\end{enumerate}

\end{document}
